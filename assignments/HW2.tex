\documentclass[12pt, letterpaper]{scrartcl}


\usepackage{fullpage} % Set margins and place page numbers at bottom center
\usepackage[shortlabels]{enumitem} % Use a. in the enumerate
\usepackage{amsmath} % aligned equations
\usepackage{graphicx} % include figure
\usepackage{float} % usage of H for figure float
\usepackage{amssymb} % \blacksqure
\usepackage{xhfill} % fill horizontal line

\usepackage{colortbl}
\usepackage{xcolor} % colors
\usepackage{sectsty} % section coloring
\usepackage{setspace}
\usepackage{bm}
\onehalfspacing
\sectionfont{\color{blue}}  % sets colour of sections

%%%%%%%%%%%%%%%%%%%%%%%%%%%%%%%%%%%%%%%%%%%%%%%%%%%%%%%%%%%%%%%%%%%%%%
% MY COMMANDS                                                        %
%%%%%%%%%%%%%%%%%%%%%%%%%%%%%%%%%%%%%%%%%%%%%%%%%%%%%%%%%%%%%%%%%%%%%%
\newcommand{\Z}{\mathbb{Z}}
\newcommand{\R}{\mathbb{R}}
\newcommand{\C}{\mathbb{C}}
\newcommand{\F}{\mathbb{F}}
\newcommand{\bigO}{\mathcal{O}}
\newcommand{\Real}{\mathcal{Re}}
\newcommand{\poly}{\mathcal{P}}
\newcommand{\mat}{\mathcal{M}}
\DeclareMathOperator{\Span}{span}
\newcommand{\Hom}{\mathcal{L}}
\DeclareMathOperator{\Null}{null}
\DeclareMathOperator{\Range}{range}
\newcommand{\defeq}{\vcentcolon=}
\newcommand{\restr}[1]{|_{#1}}
\newcommand*\diff{\mathop{}\!\mathrm{d}}


\begin{document}

% ### Header - start ###
\begin{center}
    \hrule
    \vspace{0.4cm}
    { \textbf{{\large Homework 2}} \\ MATH 564 --- Intermediate Differential Equations}
\end{center}
{ Name: \textbf{Ali Zafari} \hspace{\fill} Fall 2023 } \newline\hrule
% ### Header - end ###
% \section*{Exercises to be Considered \xrfill[2pt]{3pt}[blue]}
% \begin{itemize}[-]
%     \item 2.5.17
%     \item 2.5.20
%     \item 2.6.7
% \end{itemize}
% \vskip1mm\hrule

\section*{2.5 Linear Systems with Constant Coefficients \xrfill[2pt]{3pt}[blue]}

\subsubsection*{Exercise 2.5.15}
Calculating eigenvalues by $\det (A-\lambda I)=0$
\begin{align*}
    % \det
    % \left(\begin{array}{ccc}
    %     1-\lambda & 0 & 3 \\
    %     8 & 1-\lambda & -1 \\
    %     5 & 1 & -1-\lambda
    % \end{array}\right)=(1-\lambda)^3+24++0-15(1-\lambda)+(1-\lambda)-0=0\\
    \lambda_1, \lambda_2, \lambda_3=-3, 2+\sqrt{7}, 2-\sqrt{7}
\end{align*}
Their corresponding eigenvectors are:
\begin{align*}
    &v_1\in\left\{\alpha\left(\begin{array}{c} -3/4 \\ 7/4 \\ 1 \end{array}\right) | \alpha\in\F\right\}\\
    &v_2\in\left\{\beta\left(\begin{array}{c} -(1+\sqrt{7})/2 \\ (3\sqrt{7}+11)/2 \\ 1 \end{array}\right) | \alpha\in\F\right\}\\
    &v_3\in\left\{\gamma\left(\begin{array}{c} (-1+\sqrt{7})/2 \\ -(3\sqrt{7}-11)/2 \\ 1 \end{array}\right) | \alpha\in\F\right\}
\end{align*}
If we find the initial condition as a linear combination of those eigenvectors:
\begin{align*}
    \eta=
    \left(\begin{array}{c} 0 \\ -2 \\ -7 \end{array}\right)
    =
    \alpha\left(\begin{array}{c} -3/4 \\ 7/4 \\ 1 \end{array}\right)
    +
    \beta\left(\begin{array}{c} -(1+\sqrt{7})/2 \\ (3\sqrt{7}+11)/2 \\ 1 \end{array}\right)
    +
    \gamma\left(\begin{array}{c} (-1+\sqrt{7})/2 \\ -(3\sqrt{7}-11)/2 \\ 1 \end{array}\right)
\end{align*}
then $\alpha, \beta, \gamma$ are known.\\
Since there are 3 distinct eigenvalues, fundamental matrix will be collection of these 3 columns:
\begin{align*}
    \phi_1(t)&=e^{-3t}\alpha\left(\begin{array}{c} -3/4 \\ 7/4 \\ 1 \end{array}\right)\\
    \phi_2(t)&=e^{(2+\sqrt{7})t}\beta\left(\begin{array}{c} -(1+\sqrt{7})/2 \\ (3\sqrt{7}+11)/2 \\ 1 \end{array}\right)\\
    \phi_3(t)&=e^{(2-\sqrt{7})t}\gamma\left(\begin{array}{c} (-1+\sqrt{7})/2 \\ -(3\sqrt{7}-11)/2 \\ 1 \end{array}\right)
\end{align*}
\vskip1mm\hrule
\subsubsection*{Exercise 2.5.16}
\begin{align*}
    (A-I)^2v=
    \left(\begin{array}{cccc}
        2 & -1 & -4 & 2 \\
        2 & 2 & -2 & -4\\
        2 & -1 & -4 & 2\\
        1 & 2 & -1 & -4
    \end{array}\right)^2
    \left(\begin{array}{c} \alpha \\ \beta \\ \gamma \\ \zeta \end{array}\right)
    =
    \left(\begin{array}{cccc}
        -4 & 4 & 8 & -8 \\
        0 & -4 & 0 & 8\\
        -4 & 4 & 8 & -8\\
        0 & -4 & 0 & 8
    \end{array}\right)
    \left(\begin{array}{c} \alpha \\ \beta \\ \gamma \\ \zeta \end{array}\right)=0
\end{align*}
solving above shows that $v$ has the form of $\left(\begin{array}{c} 2\gamma \\ 2\zeta \\ \gamma \\ \zeta \end{array}\right)$. 
\begin{align*}
    (A-I)^2w=
    \left(\begin{array}{cccc}
        4 & -1 & -4 & 2 \\
        2 & 4 & -2 & -4\\
        2 & -1 & -2 & 2\\
        1 & 2 & -1 & -2
    \end{array}\right)^2
    \left(\begin{array}{c} \alpha \\ \beta \\ \gamma \\ \zeta \end{array}\right)
    =
    \left(\begin{array}{cccc}
        8 & 0 & -8 & 0 \\
        8 & 8 & -8 & -8\\
        4 & 0 & -4 & 0\\
        4 & 4 & -4 & -4
    \end{array}\right)
    \left(\begin{array}{c} \alpha \\ \beta \\ \gamma \\ \zeta \end{array}\right)=0
\end{align*}
solving above shows that $w$ has the form of $\left(\begin{array}{c} \alpha \\ \beta \\ \alpha \\ \beta \end{array}\right)$.

To find fundamental matrix, we assume a general initial condition and write it as sum of two vectors in the corresponding eigenspaces:
\begin{align*}
    \eta=\left(\begin{array}{c} \eta_1 \\ \eta_2 \\ \eta_3 \\ \eta_4 \end{array}\right)=\left(\begin{array}{c} 2(\eta_1-\eta_3) \\ 2(\eta_2-\eta_4) \\ \eta_1-\eta_3 \\ \eta_2-\eta_4 \end{array}\right)
    +
    \left(\begin{array}{c} 2\eta_3-\eta_1 \\ 2\eta_4-\eta_2 \\ 2\eta_3-\eta_1 \\ 2\eta_4-\eta_2 \end{array}\right)
    =
    v_1+w_1
\end{align*}
the solution can be written as
\begin{align*}
    \phi(t)&=e^t[I+t(A-I)]v_1+e^{-t}[I+t(A+I)]w_1\\
    =&e^t\left(\begin{array}{cccc}
        1+2t & -t & -4t & 2t \\
        2t & 1+2t & -2t & -4t\\
        2t & -t & 1-4t & 2t\\
        1t & 2t & -t & 1-4t
    \end{array}\right)\left(\begin{array}{c} 2(\eta_1-\eta_3) \\ 2(\eta_2-\eta_4) \\ \eta_1-\eta_3 \\ \eta_2-\eta_4 \end{array}\right)\\
    &+e^{-t}\left(\begin{array}{cccc}
        1+4t & -t & -4t & 2t \\
        2t & 1+4t & -2t & -4t\\
        2t & -t & 1-2t & 2t\\
        t & 2t & -t & 1-2t
    \end{array}\right)\left(\begin{array}{c} 2\eta_3-\eta_1 \\ 2\eta_4-\eta_2 \\ 2\eta_3-\eta_1 \\ 2\eta_4-\eta_2 \end{array}\right)
    \tag{$*$}
\end{align*}
By letting only one of $\eta_i=1$ at each time, we can find 4 set of solutions $\phi_1(t), \phi_2(t), \phi_3(t), \phi_4(t)$. Putting them together as columns will give us the fundamental matrix ($e^{tA}$) such that $\Phi(0)=I$, as
\begin{align*}
    \Phi(t)=\left(\begin{array}{cccc}
        2e^t-e^{-t} & -te^{-t} & -2e^t+2e^{-t} & 2te^{-t} \\
        2te^t & 2e^t-e^{-t} & -2te^t & -2e^t+2e^{-t}\\
        e^t-e^{-t} & -te^{-t} & -e^t+2e^{-t} & 2te^{-t}\\
        te^t & e^t-e^{-t} & -te^t & -e^t+2e^{-t}
    \end{array}\right)
\end{align*}
A particular solution which satisfies $\eta=\left(\begin{array}{c} 1 \\ 0 \\ -1 \\ 0 \end{array}\right)$ will be calculated from Eq. ($*$):
\begin{align*}
    \phi(t)=\left(\begin{array}{c} 4e^t-3e^{-t} \\ 4te^t \\ 2e^t-3e^{-t} \\ 2te^t \end{array}\right)
\end{align*}
% \begin{align*}
%     \eta=\left(\begin{array}{c} 1 \\ 0 \\ -1 \\ 0 \end{array}\right)=\left(\begin{array}{c} 4 \\ 0 \\ 2 \\ 0 \end{array}\right)+\left(\begin{array}{c} -3 \\ 0 \\ -3 \\ 0 \end{array}\right)
% \end{align*}
\vskip1mm\hrule
\subsubsection*{Exercise 2.5.17}
First considering the homogeneous system:
\begin{align*}
    A=
    \left(\begin{array}{cc}
        1 & 1 \\
        2 & 0 \\
    \end{array}\right)
\end{align*}
Eigenvalues are $\lambda_1=-1$ and $\lambda_2=2$, with corresponding eigenvectors $v_1=\left(\begin{array}{c}-1\\2\end{array}\right)$ and $v_2=\left(\begin{array}{c}1\\1\end{array}\right)$. Then a fundamental matrix will be:
\begin{align*}
    \Phi(t)=
    \left(\begin{array}{cc}
        -e^{-t} & e^{2t} \\
        2e^{-t} & e^{2t}
    \end{array}\right)
\end{align*}
A solution satisfying the initial condition as in $\eta=\left(\begin{array}{c}1\\1\end{array}\right)=0v_1+v_2$, will be:
\begin{align*}
    \phi_h(t)=\Phi(t)\left(\begin{array}{c}0\\1\end{array}\right)
\end{align*}
Then to find the solution of the non-homogeneous part:
\begin{align*}
    \psi(t)&=\Phi(t)\int_0^t
    \frac{-e^{-s}}{3}\left(\begin{array}{cc}
        e^{2s} & -e^{2s} \\
        -2e^{-s} & -e^{-s}
    \end{array}\right)
    \left(\begin{array}{c}\sin s\\\cos s\end{array}\right)\diff s\\
    &=\frac{1}{3}\Phi(t)\int_0^t
    \left(\begin{array}{c}
        -e^{s}\sin s+e^{s}\cos s\\
        2e^{-2s}\sin s+e^{-2s}\cos s
    \end{array}\right)\diff s\\
    &=\frac{1}{3}\Phi(t)
    \left(\begin{array}{c}
        e^{t}\cos t -1\\
        -e^{-2t}\cos t+1
    \end{array}\right)
\end{align*}

Then the solution is:
\begin{align*}
    \phi(t)=\phi_h(t)+\psi(t)
\end{align*}

\vskip1mm\hrule
\subsubsection*{Exercise 2.5.18}
Let $y_1=y$ and $y_2=y'$, then $y_2=y_1'$ and $y_2'=y''$. Then we have $y_2'=-py_2-qy_1=0$ as in:
\begin{align*}
    \left(\begin{array}{c}
        y_1' \\
        y_2'
    \end{array}\right)
    =
    \left(\begin{array}{cc}
        0 & 1 \\
        -q & -p 
    \end{array}\right)
    \left(\begin{array}{c}
        y_1 \\
        y_2
    \end{array}\right)
\end{align*}
Eigenvalues are $\lambda_{1,2}=\frac{-p\pm\sqrt{p^2-4q}}{2}$.
\vskip1mm\hrule
\subsubsection*{Exercise 2.5.19}
Existence of 2 distinct eigenvalues gives us directly the fundamental matrix:
\begin{align*}
    \Phi(t)=\left(\begin{array}{cc}
        e^{\lambda_1t}\left[\begin{array}{c}
        v_1 \\
        v_2
        \end{array}\right] &, e^{\lambda_2t}\left[\begin{array}{c}
        w_1 \\
        w_2
        \end{array}\right] \\
    \end{array}\right)
\end{align*}
\vskip1mm\hrule
\subsubsection*{Exercise 2.5.20}
$\lambda=-p/2$ with multiplicity 2. Then the solution and fundamental matrix are as follows:
\begin{align*}
    \phi(t)&=e^{-pt/2}[I+t(A+\frac{p}{2}I)]\eta\\
    &=e^{-pt/2}\left(\begin{array}{cc}
        1+\frac{p}{2}t & t \\
        -\frac{p^2}{4}t & 1-\frac{p}{2}t 
    \end{array}\right)\eta\\
    &=\Phi(t)\eta
\end{align*}
where the last line is true since $\eta$ is any vector in $\mathbb{F}^2$.\\
The general solution for $y''+py'+q=0$ is linear combination of the first row of fundamental matrix:
\begin{align*}
    \rho(t)=\alpha e^{-pt/2}(1+\frac{p}{2}t)+\beta e^{-pt/2}t
\end{align*}
\vskip1mm\hrule

\subsubsection*{Exercise 2.5.24}
\begin{align*}
    A^2&=
    \left(\begin{array}{cc}
        0 & 1 \\
        -1 & 0 \\
    \end{array}\right)
    \left(\begin{array}{cc}
        0 & 1 \\
        -1 & 0 \\
    \end{array}\right)=
    \left(\begin{array}{cc}
        -1 & 0 \\
        0 & -1 \\
    \end{array}\right)=-I\\
    A^3&=A^2A=(-I)A=-A\\
    A^4&=A^2A^2=I
\end{align*}
Therefore:
\begin{align*}
A^m=
    \begin{cases}
        (-1)^{m/2}I & \text{if m is even}\\
        (-1)^{\lfloor m/2\rfloor}A & \text{if m is odd} 
    \end{cases}
\end{align*}
\vskip1mm\hrule
\subsubsection*{Exercise 2.5.25}
\begin{align*}
    e^{tA}&=I + \sum_{m=1}^\infty\frac{t^m}{m!}A^m\\
    &=I + \sum_{n=1}^\infty\frac{t^{2n}}{(2n)!}A^{2n} + \sum_{n=1}^\infty\frac{t^{2n+1}}{(2n+1)!}A^{2n+1}\\
    &=I+\sum_{n=1}^\infty\frac{t^{2n}(-1)^n}{(2n)!}I + \sum_{n=0}^\infty\frac{t^{2n+1}(-1)^n}{(2n+1)!}A\\
    &=\left(\sum_{n=0}^\infty\frac{t^{2n}(-1)^n}{(2n)!}\right)I + \left(\sum_{n=0}^\infty\frac{t^{2n+1}(-1)^n}{(2n+1)!}\right)A\\
    &=\cos t I + \sin t A\\
    &=\left(\begin{array}{cc}
        \cos t & 0 \\
        0 & \cos t \\
    \end{array}\right)+
    \left(\begin{array}{cc}
        0 & \sin t \\
        -\sin t & 0 \\
    \end{array}\right)\\
    &=\left(\begin{array}{cc}
        \cos t & \sin t \\
        -\sin t & \cos t \\
    \end{array}\right)
\end{align*}
\vskip1mm\hrule
\subsubsection*{Exercise 2.5.27}
Let $y_1=y$ and $y_2=y'$, then $y_2=y_1'$ and $y_2'=y''$.
\begin{align*}
    \left(\begin{array}{c}
        y_1' \\
        y_2'
    \end{array}\right)
    =
    \underbrace{\left(\begin{array}{cc}
        0 & 1 \\
        -1 & 0 
    \end{array}\right)}_{A}
    \left(\begin{array}{c}
        y_1 \\
        y_2
    \end{array}\right)
    +
    \left(\begin{array}{c}
        0 \\
        f(t)
    \end{array}\right)
\end{align*}
A fundamental matrix for the homogeneous system is $e^{tA}$, which from \textbf{Exercise 2.5.24-25} we have:
\begin{align*}
    \Phi(t)=e^{tA}=\left(\begin{array}{cc}
        \cos t & \sin t \\
        -\sin t & \cos t \\
    \end{array}\right)
\end{align*}
therefore
\begin{align*}
    \phi_h(t)=\Phi(t-t_0)\eta
\end{align*}
Then to find the solution of the non-homogeneous part:
\begin{align*}
    \psi(t)&=\Phi(t)\int_{t_0}^t
    \left(\begin{array}{cc}
        \cos s & -\sin s \\
        \sin s & \cos s \\
    \end{array}\right)
    \left(\begin{array}{c}0\\f(s)\end{array}\right)\diff s\\
    &=\left(\begin{array}{cc}
        \cos t & \sin t \\
        -\sin t & \cos t \\
    \end{array}\right)
    \left(\begin{array}{c}
        \int_{t_0}^t-f(s)\sin s\diff s\\
        \int_{t_0}^tf(s)\cos s\diff s
    \end{array}\right)\\
    &=
    \left(\begin{array}{c}
        \int_{t_0}^tf(s)\sin(t-s)\diff s\\
        \int_{t_0}^tf(s)\cos(t-s)\diff s
    \end{array}\right)
\end{align*}

Then the solution is:
\begin{align*}
    \phi(t)=\phi_h(t)+\psi(t)
\end{align*}

\vskip1mm\hrule

\clearpage

\section*{2.6 Similarity of Matrices and the Jordan Canonical Form \xrfill[2pt]{3pt}[blue]}

\subsubsection*{Exercise 2.6.5}
Eigenvalues:
\begin{align*}
    (1-\lambda)(-\lambda)-2=0\\
    \lambda_1,\lambda_2=2, -1
\end{align*}
To find transformation matrix:
\begin{align*}
    \left(\begin{array}{cc}
        1 & 1 \\
        2 & 0 \\
    \end{array}\right)
    \left(\begin{array}{cc}
        t_{11} & t_{12} \\
        t_{21} & t_{22} \\
    \end{array}\right)
    &=
    \left(\begin{array}{cc}
        t_{11} & t_{12} \\
        t_{21} & t_{22} \\
    \end{array}\right)
    \left(\begin{array}{cc}
        2 & 0 \\
        0 & -1 \\
    \end{array}\right)\\
    \left(\begin{array}{cc}
        t_{11}+t_{21} & t_{12}+t_{22} \\
        2t_{11} & 2t_{12} \\
    \end{array}\right)
    &=
    \left(\begin{array}{cc}
        2t_{11} & -t_{12} \\
        2t_{21} & -t_{22} \\
    \end{array}\right)
\end{align*}
Then $T=\left(\begin{array}{cc} 1 & 1 \\ 1 & -2 \\ \end{array}\right)$.
\vskip1mm\hrule

\subsubsection*{Exercise 2.6.6}
\begin{align*}
    T^{-1}AT
    &=\frac{-1}{3}\left(\begin{array}{cc} 
    -2 & -1 \\ -1 & 1 \\ 
    \end{array}\right)
    \left(\begin{array}{cc} 
    1 & 1 \\ 2 & 0 \\ 
    \end{array}\right)
    \left(\begin{array}{cc} 
    1 & 1 \\ 1 & -2 \\ 
    \end{array}\right)\\
    &=\frac{-1}{3}\left(\begin{array}{cc} 
    -2 & -1 \\ -1 & 1 \\ 
    \end{array}\right)
    \left(\begin{array}{cc} 
    2 & -1 \\ 2 & 2 \\ 
    \end{array}\right)\\
    &=\frac{-1}{3}\left(\begin{array}{cc} 
    -6 & 0 \\ 0 & 3 \\ 
    \end{array}\right)\\
    &=\left(\begin{array}{cc}
        2 & 0 \\
        0 & -1 \\
    \end{array}\right)
\end{align*}
\vskip1mm\hrule

\subsubsection*{Exercise 2.6.7}
Since left hand side of $e^{tA}T=Te^{tJ}$ is a fundamental matrix, then $Te^{tJ}$ also forms a fundamental matrix. Therefore
\begin{align*}
    \Phi(t)=Te^{tJ}=\left(\begin{array}{cc} 
    1 & 1 \\ 1 & -2 \\ 
    \end{array}\right)
    \left(\begin{array}{cc}
        e^{2t} & 0 \\
        0 & e^{-t} \\
    \end{array}\right)
    =
    \left(\begin{array}{cc}
        e^{2t} & e^{-t} \\
        e^{2t} & -2e^{-t} \\
    \end{array}\right)
\end{align*}
\vskip1mm\hrule
\clearpage

\end{document}

